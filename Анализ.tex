\section{Анализ предметной области}
\subsection{Общественный транспорт и его роль в жизни города}

Перевозка людей на транспорте является ответственной и важной работой в жизни города. Ежедневно сотни тысяч людей используют общественный транспорт, чтобы добраться из пункта А в пункт Б. Благодаря городскому транспорту люди без труда могут доехать до места работы, учебы, медицинских учреждений или совершить поездку по своим делам. К общественному транспорту относятся:
\begin{itemize}
	\item Автобусы -- один из самых распространенных видов городского общественного транспорта. Автобусы следуют по определенным маршрутам и останавливаются на остановках для посадки и высадки пассажиров.
	\item Трамваи -- электрические транспортные средства, движущиеся по рельсам. Трамваи также следуют по установленным маршрутам.
	\item Метро -- подземный или надземный железнодорожный транспорт, обычно имеющий несколько линий и станций. Метро обеспечивает быстрое и эффективное передвижение в городе, особенно в крупных мегаполисах.
	\item Поезда -- большинство городов имеют железнодорожное сообщение, которое обеспечивает связь между ними.
	\item Троллейбусы -- электрические транспортные средства, которые движутся по улицам города, питаясь электричеством от воздушной контактной сети, установленной над дорогами. Троллейбусы также следуют по установленным маршрутам.
	\item Маршрутные такси (маршрутки) -- небольшие пассажирские автобусы или микроавтобусы, следующие по установленным маршрутам, но обычно имеющие более гибкий график и маршрут, чем обычные автобусы.
	\item Такси -- услуга пассажирского транспорта, предоставляемая частными лицами или специализированными компаниями. Такси обычно предоставляет индивидуальные или групповые поездки на короткие и средние расстояния в пределах города или за его пределами.
\end{itemize}

 Разновидностей городского транспорта много, каждый имеет свои преимущества и недостатки. Любой человек может выбрать тот вид транспорта, который ему наиболее удобен. Общественный транспорт играет важную роль в жизни города, поэтому услуги перевозки людей на транспорте должны выполняться качественно и безопасно как для самих пассажиров, так и для других участников дорожного движения города.

\subsection{Преимущества и недостатки общественного транспорта -- Автобусы}

Автобусы являются основным видом общественного транспорта в жизни города. Они имеют следующие преимущества:
\begin{itemize}
	\item Общедоступность. Автобусы предоставляют широкий охват маршрутов, что делает их доступными для большого количества жителей города. Это особенно важно для обслуживания пригородных и отдаленных районов, где другие формы общественного транспорта могут быть менее эффективны.
	\item Экономическая эффективность. Автобусы, как правило, имеют более низкие затраты на инфраструктуру по сравнению с другими формами общественного транспорта, такими как метро или трамвайные системы.
	\item Гибкость. Автобусы легче адаптировать к изменениям в маршрутах и времени движения, что делает их гибкими в управлении и обеспечении обслуживания в тех местах, где они наиболее необходимы.
	\item Интермодальность. Автобусы интегрируются в систему общественного транспорта, что позволяет пассажирам пересаживаться с одного вида транспорта на другой для удобства перемещения по городу.
	\item Экологическая устойчивость. Некоторые автобусы работают на альтернативных источниках энергии, таких как электричество или биотопливо, что делает их более экологически чистыми по сравнению с автомобилями, работающими на бензине или дизеле.
	\item Социальная интеграция. Автобусы способствуют социальной интеграции, предоставляя возможность перемещения для всех слоев населения, включая тех, кто не может себе позволить собственный автомобиль или другие виды транспорта.
\end{itemize}

У автобусов помимо преимуществ, также есть и недостатки:
\begin{itemize}
	\item Задержки. Иногда автобусы застревают в пробках, особенно в перегруженных городских центрах, что приводит к сбоям в графике движения и недовольству пассажиров.
	\item Негативное воздействие на окружающую среду и шум. Движение автобусов может вызывать загрязнение воздуха и шум, особенно при использовании дизельных двигателей, что наносит непоправимый вред экологии и здоровью пассажиров.
	\item Ограниченная вместимость. Автобусы имеют ограниченное количество сидячих мест, поэтому в периоды пиковой нагрузки они могут быть переполнены, что вызывает дискомфорт и неудобства для пассажиров.
	\item Безопасность. В сравнении с другими видами транспорта, такими как метро или поезда, автобусы могут представлять больший риск для безопасности, особенно при авариях на дорогах или в случае неправильного поведения пассажиров или других участников дорожного движения.
	\item Ограниченная скорость. Автобусы обычно движутся медленнее, чем другие виды транспорта, особенно в городах с высоким трафиком, что может сделать их менее привлекательными для тех, кто ценит скорость, удобство и время.
	\item Ограниченность маршрутов. В некоторых городах автобусные маршруты могут быть ограничены, в следствии чего, некоторые районы могут оставаться недоступными или иметь ограниченное обслуживание, особенно в ночное время или по выходным дням.
	\item Комфорт и удобство. В сравнении с более закрытыми видами транспорта, такими как метро или поезда, автобусы более подвержены воздействию внешних факторов и погодных условий, таким как пыль или грязь, что может сказаться на общей чистоте транспорта. Это может снизить общее восприятие комфорта и удобства пассажиров, особенно для тех, кто ценит чистоту и уют во время поездки.
\end{itemize}

Автобусы можно классифицировать по следующим параметрам:
\begin{itemize}
	\item Автобусы особо малого и малого класса. Автобусы малого класса, по сути, являются небольшими по размеру и предназначены для использования на городских маршрутах и в пригороде. Их можно условно разделить на две подгруппы: автобусы особо малого класса и транспортные средства (ТС) малого класса.
	Первая подгруппа, автобусы особо малого класса, часто используются в качестве маршруток и имеют ограничение в длине до 5 метров, а вместимость не превышает 10 человек (учитывая только сидячие места).
	Вторая подгруппа машин, то есть малых автобусов, имеет размеры в пределах 6-8 метров и вмещает до 40 пассажиров, включая 20 сидячих мест в салоне.
	\item Автобусы среднего класса. Автобусы среднего класса представляют собой транспортные средства длиной от 8 до 9,5 метров, способные вместить до 60 пассажиров, при этом половина из этих мест предназначена для сидения пассажиров. Среди них особенно популярны городские автобусы, которые разработаны с учетом специфики использования в городских условиях. Они обладают низким полом для быстрой посадки и высадки пассажиров, хорошей маневренностью, что позволяет им легко передвигаться по загруженным городским улицам, а также имеют большие накопительные площадки и удобные поручни. Все эти особенности делают их идеальным выбором для обслуживания городских маршрутов.
	\item Крупногабаритные автобусы. Автобусы большого класса имеют длину до 12 метров и вмещают до 90 пассажиров. Количество посадочных мест может изменяться в диапазоне от 30 до 40 в зависимости от марки и модели автобуса. Такие транспортные средства предназначены для перевозки пассажиров на большие расстояния, превышающие 500 километров и более. 
\end{itemize}

\subsection{Преимущества и недостатки общественного транспорта -- Трамваи}

Трамвай является одним из старейших видов городского пассажирского общественного транспорта из существующих в начале XXI века. Трамваи имеют следующие преимущества:
\begin{itemize}
	\item Экологическая чистота. Трамваи являются одним из самых экологически чистых видов общественного транспорта, так как они работают на электричестве, а не на нефтепродуктах, и не выделяют вредных выбросов в атмосферу.
	\item Энергоэффективность. Электрические трамваи обычно имеют более высокую энергоэффективность по сравнению с автобусами или легковыми автомобилями, особенно если электричество производится из возобновляемых источников энергии.
	\item Надежность. Трамваи имеют специальные выделенные полосы движения, что делает их менее подверженными пробкам и задержкам, связанным с автомобильным трафиком.
	\item Высокая вместимость. Трамваи способны вместить значительное количество пассажиров, особенно на протяженных маршрутах, что делает их эффективным видом общественного транспорта для городов с высокой плотностью населения.
	\item Экономическая эффективность. В долгосрочной перспективе трамваи могут оказаться экономически выгодными для городов благодаря снижению затрат на топливо и обслуживание по сравнению с автобусами или легковыми автомобилями.
\end{itemize}

К основным недостаткам трамваев можно отнести:
\begin{itemize}
	\item Ограниченная гибкость маршрутов. Трамваи привязаны к своим собственным путям, что делает их менее гибкими по сравнению с автобусами, которые могут с легкостью изменять маршруты.
	\item Зависимость от инфраструктуры. Для работы трамваев требуется специальная инфраструктура, включая провода для подачи электроэнергии и специальные пути, что может быть экономически невыгодно и труднореализуемо в уже существующих городах.
	\item Ограниченная маневренность. Трамваи не могут опережать другие транспортные средства, а также могут быть ограничены в движении, если что-то происходит на пути их следования, например авария или поломка.
	\item Ограниченные возможности для расширения: Поскольку трамваи требуют специальной инфраструктуры, их модернизация и расширение требует значительно больше времени и ресурсов по сравнению с другими видами транспорта, такими как автобусы или метро.
\end{itemize}

Классификацию трамваев можно составить по следующим критериям:
\begin{itemize}
	\item Электрические трамваи. Трамваи данного типа работают от электрической сети, подающейся по контактным проводам.
	\item Модернизированные исторические трамваи: ретро-трамваи, работающие на электричестве и восстановленные для туристических целей.
	\item Музейные трамваи: не работают в регулярном графике, а экспонируются в музеях или используются для особых событий.
\end{itemize}

\subsection{Преимущества и недостатки общественного транспорта -- Метро}

Метро - это эффективная система общественного транспорта, представляющая собой сеть подземных и наземных железных дорог, соединяющих различные части города и пригородные районы. Ниже перечислены преимущества метро как вида общественного транспорта:
\begin{itemize}
	\item Быстрая и надежная перевозка. Метро обычно предоставляет один из самых быстрых и надежных способов перемещения в городе, особенно в часы пик, когда дорожные пробки могут значительно замедлить передвижение на дорогах.
	\item Большая пассажировместимость. Метро способно перевозить огромное количество пассажиров в одном направлении за короткий промежуток времени благодаря длинным поездам и высокой частоте движения.
	\item Экономия времени. Поскольку интервалы между движением поездов метро обычно минимальны, пассажиры не тратят много времени на ожидание, а скорость движения позволяет быстро добираться до места назначения.
	\item Экологическая чистота. Метро работает на электричестве, что делает его одним из самых экологически чистых видов общественного транспорта, поскольку не создает вредных выбросов и не загрязняет окружающую среду.
	\item Эффективность в больших городах. В крупных городах с высокой плотностью населения и интенсивным транспортным движением метро является наиболее эффективным способом перевозки большого количества людей в пределах города.
\end{itemize}

У метро есть следующие недостатки:
\begin{itemize}
	\item Высокие затраты на строительство и обслуживание. Строительство и поддержание метрополитена требует значительных финансовых ресурсов, что может значительно сказаться на бюджете города.
	\item Ограниченная география. Метро ограничено городской зоной и не всегда достигает отдаленных пригородных районов, что может создавать неудобства для жителей этих районов.
	\item Ограниченные маршруты и остановки. Метро имеет фиксированные маршруты и остановки, что может быть неудобно для пассажиров, которым нужен доступ к более удаленным или малопосещаемым районам.
	\item Перегруженность в часы пик. В периоды пиковой загруженности метро может стать перегруженным, что приводит к дискомфорту и неудобствам для пассажиров, особенно в тесных вагонах.
	\item Ограниченные часы работы. В некоторых городах метро может быть закрыто ночью или иметь ограниченное время работы, что ограничивает его доступность для пассажиров в нерабочие часы.
\end{itemize}

Метро - это не только эффективный и быстрый способ перевозки в городе, но и символ современности и развития городской инфраструктуры. Его преимущества включают быструю и надежную перевозку, высокую вместимость, экологическую чистоту и экономию времени для пассажиров. Однако, существуют и недостатки, такие как высокие затраты на строительство и обслуживание, ограниченная география и перегруженность в часы пик.

В целом, метро остается одним из наиболее важных элементов городской жизни, обеспечивая удобство и доступность перемещения для жителей и посетителей городов. Его роль в улучшении качества городской среды, снижении транспортных проблем и влиянии на развитие экономики делает метро неотъемлемой частью современного городского образа.

\subsection{Преимущества и недостатки общественного транспорта -- Поезда}

Поезда - это один из наиболее распространенных и важных видов общественного транспорта, предоставляющий возможность пассажирам быстро и эффективно перемещаться на дальние расстояния как внутри страны, так и за ее пределами. Рассмотрим преимущества поездов:
\begin{itemize}
	\item Быстрая и комфортабельная перевозка. Поезда передвигаются с высокой скоростью и предоставляют пассажирам возможность комфортно провести время во время поездки, особенно на дальних маршрутах.
	\item Большая вместимость. Поезда могут перевозить большое количество пассажиров за один рейс, что делает их эффективным видом транспорта для перемещения больших групп людей или в периоды пиковой загрузки.
	\item Экологическая эффективность. Некоторые виды поездов, такие как электрические или гибридные поезда, могут быть более экологически чистыми по сравнению с другими видами транспорта, так как они не используют топливо внутреннего сгорания.
	\item Надежность и безопасность. Поезда тщательно осматриваются перед отправлением, проходят техническое обслуживание, а также регламентируются строгим правилам безопасности, что делает их одним из самых надежных видов общественного транспорта.
	\item Удобство для путешественников. В поездах часто предоставляются различные удобства для пассажиров, такие как купе, рестораны, бары, Wi-Fi и развлекательные программы, что делает поездки более приятными и комфортными.
\end{itemize}

К недостаткам железнодорожного транспорта можно отнести:
\begin{itemize}
	\item Ограниченная география. Поезда могут не достигать всех районов и населенных пунктов, особенно в отдаленных и малонаселенных областях, что ограничивает их доступность.
	\item Высокая стоимость. Цена билета на поезд может быть выше, по сравнению с другими видами общественного транспорта, особенно на дальние расстояния или на поездах с высоким уровнем комфорта и услуг.
	\item Ограниченное расписание. Поезда следуют строгому расписанию, что может быть неудобно для пассажиров, нуждающихся в гибкости во времени отправления и прибытия.
	\item Перегруженность в периоды пиковой загрузки. В дни высокого спроса на билеты, такие как праздничные и выходные дни, поезда могут быть переполнены, что создает ограниченность для пассажиров.
\end{itemize}

Роль поездов как вида общественного транспорта в жизни города огромна и многоаспектна. Вот несколько ключевых аспектов их влияния:
\begin{itemize}
	\item Связь и мобильность. Поезда обеспечивают жителей города и его посетителей быстрой и удобной связью между различными районами и населенными пунктами. Они позволяют людям легко перемещаться по городу и за его пределами, обеспечивая мобильность и доступность для всех слоев населения.
	\item Развитие инфраструктуры. Строительство и поддержание железнодорожных путей, станций и инфраструктуры поездов способствует развитию городской инфраструктуры в целом. Это включает в себя создание новых рабочих мест, инвестиции в градостроительство и улучшение условий жизни в городе.
	\item Социальная интеграция. Поезда создают возможности для социальной интеграции и обмена культурными и экономическими идеями между различными сообществами и группами населения. Они предоставляют пассажирам возможность встречаться, общаться и делиться опытом, способствуя культурному разнообразию и взаимопониманию.
\end{itemize}

В целом, поезда играют важную роль в жизни города, обеспечивая связь, мобильность, развитие экономики и социальную интеграцию. Их значимость в общественном транспорте подчеркивается их влиянием на различные аспекты городской жизни и развития.
	
\subsection{Преимущества и недостатки общественного транспорта -- Троллейбусы}

Троллейбусы - это вид общественного транспорта, который работает на электричестве и движется по маршрутам, оборудованным специальными проводами для подачи электроэнергии. Троллейбусы имеют следующие преимущества:
\begin{itemize}
	\item Экологическая чистота. Троллейбусы работают на электричестве, что делает их одним из самых экологически чистых видов общественного транспорта. Они не выделяют вредных выбросов в атмосферу, что способствует улучшению качества воздуха в городе и снижению загрязнения окружающей среды.
	\item Эффективность в городском трафике. Троллейбусы обычно имеют отдельные полосы движения или право приоритетного проезда на дорогах, что позволяет им обходить пробки и двигаться более эффективно в городском трафике.
	\item Низкая стоимость эксплуатации. Эксплуатация троллейбусов часто обходится дешевле, чем у автобусов с двигателями внутреннего сгорания, так как электричество является более дешевым и экономически выгодным источником энергии.
	\item Тихий ход.Троллейбусы обычно имеют более тихий ход по сравнению с автобусами с двигателями внутреннего сгорания, что делает их более приятными для пассажиров и жителей города, особенно в ночное время.
\end{itemize}

К недостаткам троллейбуса, как вида общественного транспорта можно отнести:
\begin{itemize}
	\item Ограниченная мобильность.Троллейбусы зависят от наличия специальной инфраструктуры в виде проводов для передачи электроэнергии. Это ограничивает их мобильность и маршруты, так как они могут двигаться только по установленным маршрутам.
	\item Зависимость от энергосистемы. Работа троллейбусов зависит от непрерывного функционирования электросети. Проблемы с электроснабжением или обрыв проводов могут привести к остановке движения троллейбусов и создать неудобства для пассажиров.
	\item Ограниченная скорость. В некоторых случаях троллейбусы могут иметь ограниченную скорость движения из-за характеристик электрического двигателя, особенно на возвышенных участках маршрута или при больших нагрузках.
	\item Сложности в маневрировании. Из-за необходимости следовать за проводами троллейбусы могут иметь ограниченные возможности для маневрирования на дорогах, особенно в условиях ограниченного пространства или на узких улицах.
\end{itemize}

В целом, троллейбусы представляют собой экологически чистый и эффективный вид общественного транспорта, но они также имеют свои ограничения, которые следует учитывать при планировании их использования в городской транспортной системе. Развитие сети троллейбусов требует создания и поддержания специализированной инфраструктуры, включая контактные сети и депо. Это стимулирует развитие городской транспортной инфраструктуры в целом, улучшая качество дорог и общее состояние городских улиц. В заключение, троллейбусы играют важную роль в жизни города, обеспечивая экологически чистый, экономически выгодный и социально значимый вид транспорта. Их интеграция в городскую транспортную систему способствует улучшению качества жизни горожан и устойчивому развитию городских территорий.

\subsection{Преимущества и недостатки общественного транспорта -- Маршрутные такси}

Маршрутное такси (маршрутка) играет важную роль в системе общественного транспорта многих городов, предоставляя жителям гибкий и быстрый способ передвижения. Далее рассматривается преимущества и недостатки маршрутного такси, с точки зрения вида общественного транспорта, а также его роль в жизни города.
Преимущества маршрутных такси:
\begin{itemize}
	\item Высокая частота движения. Маршрутные такси часто курсируют с очень короткими интервалами, что делает их удобными для пассажиров, которым не нужно долго ждать следующего транспортного средства.
	\item Гибкость маршрутов и остановок. В отличие от автобусов, маршрутные такси могут останавливаться по запросу пассажиров в удобных для них местах, что увеличивает их привлекательность и удобство использования.
	\item Быстрая адаптация к изменяющимся условиям. Маршрутные такси могут быстро изменять свои маршруты в ответ на изменения дорожной ситуации, пробки или запросы пассажиров, что делает их более гибкими по сравнению с другими видами транспорта.
	\item Меньшие размеры и маневренность. Компактные размеры маршрутных такси позволяют им быстрее передвигаться по узким улицам и в условиях плотного городского трафика.
\end{itemize}

Недостатки маршрутных такси:
\begin{itemize}
	\item Перегруженность. В часы пик маршрутные такси могут быть сильно перегружены, что снижает комфорт для пассажиров и может вызывать задержки.
	\item Безопасность. Маршрутные такси часто менее комфортны и безопасны по сравнению с автобусами и поездами. Пассажиры могут испытывать дискомфорт из-за тесноты, отсутствия кондиционирования и неудобных сидений.
	\item Экологическое воздействие. Маршрутные такси, особенно старые модели, могут быть менее экологичными по сравнению с новыми автобусами и электромобилями, выделяя больше вредных выбросов в атмосферу.
	\item Проблемы с управлением и регулированием. В некоторых городах маршрутные такси могут работать нелегально или полулегально, что затрудняет их регулирование и контроль за качеством услуг, безопасностью и соблюдением правил дорожного движения.
\end{itemize}

Маршрутное такси играет важную роль в системе общественного транспорта города, предоставляя гибкий и быстрый способ передвижения для жителей и гостей. Маршрутные такси часто обслуживают маршруты, которые не покрываются основными видами общественного транспорта, такими как автобусы, трамваи или метро, тем самым закрывая возможные пробелы в сети общественного транспорта, обеспечивая удобство и доступность. Однако маршрутные такси также имеют свои недостатки, включая перегруженность, проблемы с безопасностью и комфортом, а также экологические и регуляторные вызовы. Для эффективного функционирования маршрутных такси необходимо сбалансированное управление и интеграция с другими видами общественного транспорта.

\subsection{Преимущества и недостатки общественного транспорта -- Такси}

Такси является важной частью городской транспортной системы. Оно обеспечивает быстрый, удобный и относительно гибкий способ передвижения, который дополняет другие виды общественного транспорта. Такси имеет следующие преимущества:
\begin{itemize}
	\item Дополнение к общественному транспорту. Такси служат дополнением к традиционным видам общественного транспорта, таким как автобусы, трамваи и метро. Оно обеспечивает перевозки в те районы и в то время, когда другие виды транспорта недоступны или менее удобны.
	\item Доступность в любое время. Такси предоставляет возможность передвижения круглосуточно, что особенно важно для людей, которым необходимо передвигаться поздно ночью или рано утром, когда общественный транспорт может не работать.
	\item Персонализированные услуги. В отличие от других видов общественного транспорта, такси предлагает персонализированные услуги, включая поездки "от двери до двери", что обеспечивает высокий уровень удобства для пассажиров.
	\item Быстрое реагирование на спрос. С помощью мобильных приложений и диспетчерских служб, такси зачастую быстрее реагирует на вызовы пассажиров, тем самым предоставляя услуги в кратчайшие сроки.
	\item Гибкость маршрутов. В отличие от автобусов и поездов, такси могут быстро адаптироваться к изменениям в дорожной ситуации и выбирать наилучший маршрут для минимизации времени в пути.
\end{itemize}

К недостаткам такси можно отнести:
\begin{itemize}
	\item Высокая стоимость. Поездки на такси, как правило, дороже, чем использование других видов общественного транспорта, что может быть существенным недостатком для людей с ограниченным бюджетом.
	\item Безопасность и надежность. Уровень безопасности и надежности может варьироваться в зависимости от компании и водителя, и некоторые пассажиры могут сталкиваться с проблемами, связанными с неадекватным поведением водителей или ненадлежащим состоянием автомобиля.
	\item Регулирование и стандарты. Неполное регулирование такси может привести к проблемам с качеством обслуживания, соблюдением правил дорожного движения и безопасности пассажиров.
\end{itemize}

Такси играет важную роль в транспортной системе города, предоставляя гибкий, удобный и персонализированный способ передвижения. Оно является важным дополнением к другим видам общественного транспорта, особенно в тех случаях, когда необходима быстрая и прямая поездка. Однако такси также имеет свои недостатки, включая высокую стоимость, воздействие на дорожное движение и экологию, а также вариативность в качестве и безопасности услуг. Для эффективного функционирования данного вида транспорта необходимо сбалансированное управление и регулирование, чтобы максимизировать их пользу и минимизировать негативные последствия для городской среды.