\section{Анализ предметной области}
\subsection{Общественный транспорт и его роль в жизни города}

Перевозка людей на транспорте является ответственной и важной работой в жизни города. Ежедневно сотни тысяч людей используют общественный транспорт, чтобы добраться из пункта А в пункт Б. Благодаря городскому транспорту люди без труда могут доехать до места работы/учебы, медицинских учреждений или совершить поездку по своим делам. К общественному транспорту относятся:
\begin{itemize}
	\item Автобусы -- один из самых распространенных видов городского транспорта. Автобусы следуют по определенным маршрутам и останавливаются на остановках для посадки и высадки пассажиров.
	\item Трамваи -- электрические транспортные средства, движущиеся по рельсам. Трамваи также следуют по установленным маршрутам.
	\item Метро -- подземный или надземный железнодорожный транспорт, обычно имеющий несколько линий и станций. Метро обеспечивает быстрое и эффективное передвижение в городе, особенно в крупных мегаполисах.
	\item Поезда -- большинство городов имеют железнодорожное сообщение, которое обеспечивает связь между ними.
	\item Троллейбусы -- электрические транспортные средства, которые движутся по улицам города, питаясь электричеством от воздушной контактной сети, установленной над дорогами. Троллейбусы также следуют по установленным маршрутам.
	\item Маршрутные такси (маршрутки) -- небольшие пассажирские автобусы или микроавтобусы, следующие по установленным маршрутам, но обычно имеющие более гибкий график и маршрут, чем обычные автобусы.
	\item Такси -- услуга пассажирского транспорта, предоставляемая частными лицами или специализированными компаниями. Такси обычно предоставляет индивидуальные или групповые поездки на короткие и средние расстояния в пределах города или за его пределами.
\end{itemize}

 Разновидностей городского транспорта много, каждый имеет свои преимущества и недостатки. Любой человек может выбрать тот вид транспорта, который ему наиболее удобен. Общественный транспорт играет важную роль в жизни города, поэтому услуги перевозки людей на транспорте должны выполняться качественно и безопасно как для самих пассажиров, так и для людей находящихся рядом с ним.

\subsection{Преимущества и недостатки общественного транспорта -- Автобусы}

Автобусы являются основным видом общественного транспорта в жизни города. Они имеют следующие преимущества:
\begin{itemize}
	\item Общедоступность. Автобусы предоставляют широкий охват маршрутов, что делает их доступными для большого количества жителей города. Это особенно важно для обслуживания пригородных и отдаленных районов, где другие формы общественного транспорта могут быть менее эффективны.
	\item Экономическая эффективность. Автобусы, как правило, имеют более низкие затраты на инфраструктуру по сравнению с другими формами общественного транспорта, такими как метро или трамвайные системы.
	\item Гибкость. Автобусы легче адаптировать к изменениям в маршрутах и времени движения, что делает их гибкими в управлении и обеспечении обслуживания в тех местах, где они наиболее необходимы.
	\item Интермодальность. Автобусы интегрируются в систему общественного транспорта, что позволяет пассажирам пересаживаться с одного вида транспорта на другой для удобства перемещения по городу.
	\item Экологическая устойчивость: Некоторые автобусы работают на альтернативных источниках энергии, таких как электричество или биотопливо, что делает их более экологически чистыми по сравнению с автомобилями, работающими на бензине или дизеле.
	\item
\end{itemize}

\subsection{Преимущества и недостатки общественного транспорта -- Трамваи}


\subsection{Преимущества и недостатки общественного транспорта -- Метро}


\subsection{Преимущества и недостатки общественного транспорта -- Поезда}


\subsection{Преимущества и недостатки общественного транспорта -- Троллейбусы}


\subsection{Преимущества и недостатки общественного транспорта -- Маршрутные такси}


\subsection{Преимущества и недостатки общественного транспорта -- Такси}


\subsection{Аддитивные технологии, их классификация}

Основное преимущество АТ состоит в том, что прототип создается за один прием, а исходными данными для него служит геометрическая модель детали. В итоге отпадает необходимость в планировании последовательности технологических процессов, специальном оборудовании для обработки материалов, транспортировке от станка к станку и т. д.

Экструзионная печать. Включает такие методы, как послойное наплавление и многоструйная печать.

Стереолитография. Стереолитографические принтеры используют специальные жидкие материалы, называемые "<фотополимерными смолами">.

Ламинирование. Слои материала наклеиваются друг на друга и обрезаются по контурам цифровой модели с помощью лазера или лезвия. 
